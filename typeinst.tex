
%%%%%%%%%%%%%%%%%%%%%%% file typeinst.tex %%%%%%%%%%%%%%%%%%%%%%%%%
%
% This is the LaTeX source for the instructions to authors using
% the LaTeX document class 'llncs.cls' for contributions to
% the Lecture Notes in Computer Sciences series.
% http://www.springer.com/lncs       Springer Heidelberg 2006/05/04
%
% It may be used as a template for your own input - copy it
% to a new file with a new name and use it as the basis
% for your article.
%
% NB: the document class 'llncs' has its own and detailed documentation, see
% ftp://ftp.springer.de/data/pubftp/pub/tex/latex/llncs/latex2e/llncsdoc.pdf
%
%%%%%%%%%%%%%%%%%%%%%%%%%%%%%%%%%%%%%%%%%%%%%%%%%%%%%%%%%%%%%%%%%%%


\documentclass[runningheads,a4paper]{llncs}

\usepackage{amssymb}
\setcounter{tocdepth}{3}
\usepackage{graphicx}

\usepackage{url}
\urldef{\mailsa}\path|{alfred.hofmann, ursula.barth, ingrid.haas, frank.holzwarth,|
\urldef{\mailsb}\path|anna.kramer, leonie.kunz, christine.reiss, nicole.sator,|
\urldef{\mailsc}\path|erika.siebert-cole, peter.strasser, lncs}@springer.com|    
\newcommand{\keywords}[1]{\par\addvspace\baselineskip
\noindent\keywordname\enspace\ignorespaces#1}

\begin{document}

\mainmatter  % start of an individual contribution

% first the title is needed
\title{State-Of-The-Art Music Recommendation}

% a short form should be given in case it is too long for the running head
%\titlerunning{}

% the name(s) of the author(s) follow(s) next
%
% NB: Chinese authors should write their first names(s) in front of
% their surnames. This ensures that the names appear correctly in
% the running heads and the author index.
%
\author{Alfred Hofmann%
\thanks{Please note that the LNCS Editorial assumes that all authors have used
the western naming convention, with given names preceding surnames. This determines
the structure of the names in the running heads and the author index.}%
\and Ursula Barth\and Ingrid Haas\and Frank Holzwarth\and\\
Anna Kramer\and Leonie Kunz\and Christine Rei\ss\and\\
Nicole Sator\and Erika Siebert-Cole\and Peter Stra\ss er}
%
\authorrunning{Lecture Notes in Computer Science: Authors' Instructions}
% (feature abused for this document to repeat the title also on left hand pages)

% the affiliations are given next; don't give your e-mail address
% unless you accept that it will be published
\institute{Springer-Verlag, Computer Science Editorial,\\
Tiergartenstr. 17, 69121 Heidelberg, Germany\\
\mailsa\\
\mailsb\\
\mailsc\\
\url{http://www.springer.com/lncs}}

%
% NB: a more complex sample for affiliations and the mapping to the
% corresponding authors can be found in the file "llncs.dem"
% (search for the string "\mainmatter" where a contribution starts).
% "llncs.dem" accompanies the document class "llncs.cls".
%

\toctitle{Lecture Notes in Computer Science}
\tocauthor{Authors' Instructions}
\maketitle


\begin{abstract}
The abstract should summarize the contents of the paper and should
contain at least 70 and at most 150 words. It should be written using the
\emph{abstract} environment.
\keywords{We would like to encourage you to list your keywords within
the abstract section}
\end{abstract}

\section{Papers References}
\cite{wang2020came}
\cite{la2022music}


\section{Introduction}

\section{State-of-the-art}
In 2020 Melchiorre, Zangerle \textit{et al.} proposed their work were they analyzed the personality bias of common algorithms in the area of music recommendation.
They stated that other work already proved that there is for example a bias on item or user level in multimedia recommendations. 
Addtionally there exists work proving a correlation between music preferences and personality traits.\\
For their experiment they created a novel dataset combining user's personality traits and music consumption.
To gather this data they used Twitter microblog posts with certain hashtags as well as the music metadata dataset MusicBrainz. 
Using the \textit{IBM Personality Insight API} the authors were able to integrate the user's personality traits according to the \textit{OCEAN} model of psychology.
The authors compared \textit{Recall@K} and \textit{NDCG@K} (normalized discounted cummulative gain), with K being the length of the recommendation list, of three popular music recommenders namely \textit{SLIM}, \textit{EASE} and \textit{Mult-VAE}.
They were able to prove that low scores on personality traits such as \textit{openness}, \textit{extraversion} and \textit{conscientiousness} lead to higher performance
while low scores for \textit{neuroticism} and \textit{agreeableness} lead to low performance according to the chosen metrics. 
Finally they stated some limitations of their work as well as planned extensions. \cite{melchiorre2020personality}\\
\\
In their work Niyazov, Mikhailova \textit{et al.} proposed a novel approach for content-based music recommendation. 
They want to recommend suitable music to users based on the acoustic similarity of songs.
This means that they use the acoustig signal of a song transformed into a suitable representation. 
To find the best approach they performed two seperate experiments and afterwards compared the quantitative results.\\
The first experiment is based on acoustic feature analysis. 
They extract acoustic features from songs, transform them into a vector representation and use 10 Nearest Neighbors with Euclidean distance 
to get the 10 best recommendations per song. \\
The second experiment is using artificial neural networks. 
The networks are trained using spectogram representations of the acoustic signals.
Using the model a vector representation for each song is predicted. 
Similar to the first approach using 10-NN the best 10 recommendations per song are gathered. \\
Both approaches got compared with random recommendations using precision, recall, F1-score and a version of the F1-score with a threshold of 0.4.
This means that all recommendations below this value are considered as irrelevant. The value was introduced by the authors based on their evaluation.\\
Comparing the quantitative measurements the authors were able to conclude that both approaches outperformed the random recommendation
whit the second approach using Neural Networks achieving the best results. \\
Addtionally the authors invesitaged genre-specific clustering via visualizing vector spaces on a plane. 
Due to genre-specific clustering being closely related to the acoustic sound of songs this again highlighted their findings that approaches 
using ANN achieve better results. 

\cite{niyazov2021content}

\section{Critical discussion and evaluation}

- personality trait -> nur twitter data considered -> gilt wsl net fia ganze population
- incomplete data (net zu allem access)
- eher wie user sein wollen (was sie posten) als was sie wirklöich anhean -> kann also von user verfälscht sein 

- some papers (e.g. niyazov) nur quantitative evaluation und eher poor results 
- only public dataset not perfect passend 
- nehmen nur bruchtteil vom dataset und assumen dass genug is = net getestet und geproved 




\section{Conclusions}

%\subsection{Figures}

%For \LaTeX\ users, we recommend using the \emph{graphics} or \emph{graphicx}
%package and the \verb+\includegraphics+ command.

%Please check that the lines in line drawings are not
%interrupted and are of a constant width. Grids and details within the
%figures must be clearly legible and may not be written one on top of
%the other. Line drawings should have a resolution of at least 800 dpi
%(preferably 1200 dpi). The lettering in figures should have a height of
%2~mm (10-point type). Figures should be numbered and should have a
%caption which should always be positioned \emph{under} the figures, in
%contrast to the caption belonging to a table, which should always appear
%\emph{above} the table; this is simply achieved as matter of sequence in
%your source.

%Please center the figures or your tabular material by using the \verb+\centering+
%declaration. Short captions are centered by default between the margins
%and typeset in 9-point type (Fig.~\ref{fig:example} shows an example).
%The distance between text and figure is preset to be about 8~mm, the
%distance between figure and caption about 6~mm.

%If screenshots are necessary, please make sure that you are happy with
%the print quality before you send the files.
%\begin{figure}
%\centering
%\includegraphics[height=6.2cm]{eijkel2}
%\caption{One kernel at $x_s$ (\emph{dotted kernel}) or two kernels at
%$x_i$ and $x_j$ (\textit{left and right}) lead to the same summed estimate
%at $x_s$. This shows a figure consisting of different types of
%lines. Elements of the figure described in the caption should be set in
%italics, in parentheses, as shown in this sample caption.}
%\label{fig:example}
%\end{figure}

%Please define figures (and tables) as floating objects. Please avoid
%using optional location parameters like ``\verb+[h]+" for ``here".

%\paragraph{Remark 1.}

%In the printed volumes, illustrations are generally black and white
%(halftones), and only in exceptional cases, and if the author is
%prepared to cover the extra cost for color reproduction, are colored
%pictures accepted. Colored pictures are welcome in the electronic
%version free of charge. If you send colored figures that are to be
%printed in black and white, please make sure that they really are
%legible in black and white. Some colors as well as the contrast of
%converted colors show up very poorly when printed in black and white.

%\subsection{Formulas}

%Displayed equations or formulas are centered and set on a separate
%line (with an extra line or halfline space above and below). Displayed
%expressions should be numbered for reference. The numbers should be
%consecutive within each section or within the contribution,
%with numbers enclosed in parentheses and set on the right margin --
%which is the default if you use the \emph{equation} environment, e.g.,
%\begin{equation}
%  \psi (u) = \int_{o}^{T} \left[\frac{1}{2}
%  \left(\Lambda_{o}^{-1} u,u\right) + N^{\ast} (-u)\right] dt \;  .
%\end{equation}

%Equations should be punctuated in the same way as ordinary
%text but with a small space before the end punctuation mark.

%\subsection{Footnotes}

%The superscript numeral used to refer to a footnote appears in the text
%either directly after the word to be discussed or -- in relation to a
%phrase or a sentence -- following the punctuation sign (comma,
%semicolon, or period). Footnotes should appear at the bottom of
%the
%normal text area, with a line of about 2~cm set
%immediately above them.\footnote{The footnote numeral is set flush left
%and the text follows with the usual word spacing.}

%\subsection{Program Code}

%Program listings or program commands in the text are normally set in
%typewriter font, e.g., CMTT10 or Courier.

%\medskip

%\noindent
%{\it Example of a Computer Program}
%\begin{verbatim}
%program Inflation (Output)
%  {Assuming annual inflation rates of 7%, 8%, and 10%,...
%   years};
%   const
%     MaxYears = 10;
%   var
%     Year: 0..MaxYears;
%     Factor1, Factor2, Factor3: Real;
%   begin
%     Year := 0;
%     Factor1 := 1.0; Factor2 := 1.0; Factor3 := 1.0;
%     WriteLn('Year  7% 8% 10%'); WriteLn;
%     repeat
%       Year := Year + 1;
%       Factor1 := Factor1 * 1.07;
%       Factor2 := Factor2 * 1.08;
%       Factor3 := Factor3 * 1.10;
%       WriteLn(Year:5,Factor1:7:3,Factor2:7:3,Factor3:7:3)
%     until Year = MaxYears
%end.
%\end{verbatim}
%
%\noindent
%{\small (Example from Jensen K., Wirth N. (1991) Pascal user manual and
%report. Springer, New York)}

%\subsection{Citations}

%For citations in the text please use
%%square brackets and consecutive numbers: \cite{jour}, \cite{lncschap},
%\cite{proceeding1} -- provided automatically
%by \LaTeX 's \verb|\cite| \dots\verb|\bibitem| mechanism.

%\subsection{Page Numbering and Running Heads}

%There is no need to include page numbers. If your paper title is too
%long to serve as a running head, it will be shortened. Your suggestion
%as to how to shorten it would be most welcome.

%\section{BibTeX Entries}

%The correct BibTeX entries for the Lecture Notes in Computer Science
%volumes can be found at the following Website shortly after the
%publication of the book:
%\url{http://www.informatik.uni-trier.de/~ley/db/journals/lncs.html}

%\subsubsection*{Acknowledgments.} The heading should be treated as a
%subsubsection heading and should not be assigned a number.

%\section{The References Section}\label{references}

%In order to permit cross referencing within LNCS-Online, and eventually
%between different publishers and their online databases, LNCS will,
%from now on, be standardizing the format of the references. This new
%feature will increase the visibility of publications and facilitate
%academic research considerably. Please base your references on the
%examples below. References that don't adhere to this style will be
%reformatted by Springer. You should therefore check your references
%thoroughly when you receive the final pdf of your paper.
%The reference section must be complete. You may not omit references.
%Instructions as to where to find a fuller version of the references are
%not permissible.

%The following section shows a sample reference list with entries for
%journal articles \cite{jour}, an LNCS chapter \cite{lncschap}, a book
%\cite{book}, proceedings without editors \cite{proceeding1} and
%\cite{proceeding2}, as well as a URL \cite{url}.
%Please note that proceedings published in LNCS are not cited with their
%full titles, but with their acronyms!

\begin{thebibliography}{4}

\bibitem{wang2020came} Wang, Dongjing and Zhang, Xin and Yu, Dongjin and Xu, Guandong and Deng, Shuiguang: 
Came: Content-and context-aware music embedding for recommendation. In: IEEE Transactions on Neural Networks and Learning Systems,
pp. 1375--1388 (2020)

\bibitem{la2022music} La Gatta, Valerio and Moscato, Vincenzo and Pennone, Mirko and Postiglione, Marco and Sperl{\'\i}, Giancarlo:
Music recommendation via hypergraph embedding. In: IEEE transactions on neural networks and learning systems (2022)

\bibitem{niyazov2021content} Niyazov, Aldiyar and Mikhailova, Elena and Egorova, Olga:
Content-based music recommendation system. In: 2021 29th Conference of Open Innovations Association (FRUCT),
pp. 274--279 (2021)

\bibitem{melchiorre2020personality} Melchiorre, Alessandro B and Zangerle, Eva and Schedl, Markus:
Personality bias of music recommendation algorithms. In: Proceedings of the 14th ACM Conference on Recommender Systems, 
pp. 533--538 (2020)

%\bibitem{jour} Smith, T.F., Waterman, M.S.: Identification of Common Molecular
%Subsequences. J. Mol. Biol. 147, 195--197 (1981)

%\bibitem{lncschap} May, P., Ehrlich, H.C., Steinke, T.: ZIB Structure Prediction Pipeline:
%Composing a Complex Biological Workflow through Web Services. In: Nagel,
%W.E., Walter, W.V., Lehner, W. (eds.) Euro-Par 2006. LNCS, vol. 4128,
%pp. 1148--1158. Springer, Heidelberg (2006)

%\bibitem{book} Foster, I., Kesselman, C.: The Grid: Blueprint for a New Computing
%Infrastructure. Morgan Kaufmann, San Francisco (1999)

%\bibitem{proceeding1} Czajkowski, K., Fitzgerald, S., Foster, I., Kesselman, C.: Grid
%Information Services for Distributed Resource Sharing. In: 10th IEEE
%International Symposium on High Performance Distributed Computing, pp.
%181--184. IEEE Press, New York (2001)

%\bibitem{proceeding2} Foster, I., Kesselman, C., Nick, J., Tuecke, S.: The Physiology of the
%Grid: an Open Grid Services Architecture for Distributed Systems
%Integration. Technical report, Global Grid Forum (2002)

%\bibitem{url} National Center for Biotechnology Information, \url{http://www.ncbi.nlm.nih.gov}

\end{thebibliography}

\end{document}
